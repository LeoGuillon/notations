\documentclass[print]{atomathematyk}

\title{Notations mathématiques}
\author{Léo Guillon}
\date{}

\begin{document}
\maketitle

\begin{abstract}
  Ce document a pour but de regrouper l’ensemble de mes «\,règles\,» de notations et d’usages de symboles mathématiques. À noter que ce document fait l’usage extensif de mes commandes personnalisées présente dans le package \texttt{maquereaux}, disponible \href{https://github.com/LeoGuillon/maquereaux}{ici}. Le rendu est généré avec ma classe personnalisée \texttt{atomathematyk}, disponible \href{https://github.com/LeoGuillon/atomathematyk}{ici}.
\end{abstract}

\tableofcontents

\section{Théorie des ensembles}\label{sec:set-theory}

\section{Algèbre}\label{sec:algebra}

\subsection{Algèbre générale}\label{sec:}

\subsection{Algèbre linéaire}\label{sec:}

\section{Analyse}\label{sec:analyse}

\section{Probabilités}\label{sec:probas}

\section{Géométrie}\label{sec:geometry}

\end{document}
